\documentclass{article}

\title{Spanish Lorca notes}
\author{Adam Martinez}
\date{}

\begin{document}

\maketitle

\section*{General information}

\subsection*{'27 generation}

Young poets educated in both old and new works.

From them stems the deep-rooted tradition of Spanish poetry, and against the
non-deep-rooted tradition they fought.

Their interest for tradition stems from two authors. Gongora and Lope de Vega.

\begin{itemize}
    \item Gongora
    \begin{itemize}
        \item Guillen and Alonso understand this poetry.
        \item They think it similar, in terms of hermetism, to the pure poetry
        of Jimenez.
        \item Happy 20s.
    \end{itemize}
    \item Lope de Vega
    \begin{itemize}
        \item Appears in commemoration of its 300th anniversary.
        \item Its poetry takes on a social role as it is a time of revolts.
        \item Fascist 30s.
    \end{itemize}
\end{itemize}

\subsubsection*{Characteristics}

\begin{itemize}
    \item Mixing of educated and popular tradition.
    \item Use of tradition and vanguard.
    \item Balance between intellectual and balanced poetry.
    \item Liricysm prevails as a formal element, to evolve into a message nearer
    to the people through epic vehemence.
\end{itemize}

\end{document}